% Preamble
\documentclass{juniorlabs}

\title{A Very Official and Pretentious Title that Sums Up Your Paper}
% Adjust vspace in juniorlabs.cls for more space in multiline titles
\subtitle{Prepared for Submission for \{insert class\}}
\author[1]{Author 1}
\author[2]{Author 2}
\affil[1]{University 1, Address 1, City, Country}
\affil[2]{University 2, Address 2, City, Country}

\email{example@unm.edu} % Include the email for the main point of contact for the report.
\date{\today}

% Document
\begin{document}
    \ReportHeading

% Abstract (if needed)
    \begin{abstract}
        A required 200–-250 word abstract starts on this line.
        Please insert here your abstract taking into account any comments from the reviewers.
        The abstract must be consistent with the contents of this paper.

        HINT: Write this section LAST!
    \end{abstract}

    \begin{keywords}
        A couple of keywords (3-5).
    \end{keywords}

% Table of Contents (if needed)
%\tableofcontents

% List of Figures (if needed)
%\listoffigures

% List of Tables (if needed)
%\listoftables

% Start your content here

    \section*{Overview of Template -- DELETE THIS FOR YOUR REPORT}
    Welcome to the amazing world of \LaTeX!
    I hope you find this outline useful and easy to use for you lab reports.
    A couple of notes to the user:
    \begin{itemize}
        \item This outline is meant to be just that: an outline. A report will not write itself, and \LaTeX will be a struggle at first to those who have not used it before. Please plan your time accordingly.
        \item The bibliography is automatically imported to this document. All you need to do is make sure you are calling the correct document name. I would recommend using a citation generator like \url{https://www.mybib.com/} or Zotero to create your BibTeX entries.
        \item Your bibliography will \emph{NOT} automatically populate--you must reference each entry in your paper at some point with \verb|\cite{citation name}|.
    \end{itemize}

    Example: \\
    \textit{As clear to even the most casual observer, these findings are without fault}~\cite{EXAMPLE}.

    Additional features such as footnotes\footnote{Which can be created like this.} and appendices~\ref{sec:-first-appendix} can be seen throughout this file.

    Equations should be added as such:
    \begin{equation}
        \frac{1}{r}\frac{\partial}{\partial r}\left(r\frac{\partial\phi}{\partial r}\right) + \frac{1}{r^2}\frac{\partial^2\phi}{\partial\theta^2} + \frac{\partial^2\phi}{\partial z^2} - \frac{\nu\sigma_f}{v}\phi = \frac{1}{v}\frac{\partial\phi}{\partial t}
    \end{equation}
    Use \verb|$$| for small bits of information in-text, such as $a=3$.
    Use \verb|\nicefrac| to do inline equations: $x = \nicefrac{5}{7}$

    Code will likely also need to be included in your report, and is traditionally included in an appendix.
    See Appendix~\ref{sec:minted-code} for an example of how to include code.
    Note that the syntax highlighting can be used on more than just Python, so be sure to hightlight with the correct language.
    See \url{https://www.overleaf.com/learn/latex/Code_Highlighting_with_minted} for more information.

    Please remember the most important thing about \LaTeX and college in general: \emph{Google is your friend.}
    This template gives a general outline for what \emph{I} think is very important to a report, but this is \emph{your} document -- please feel free to adjust the sections to whatever you feel is best for you.


    \section{Introduction} \label{introduction}
    Things to consider:
    \begin{itemize}
        \item What is the topic?
        \item Start broad, then narrow down.
        \item How is this work relevant?
        \item What was goal of the experiment?
    \end{itemize}


    \section{Theory} \label{theory}
    \begin{itemize}
        \item All general equations.
        \item How would you analyze your data?
        \item Could I read this paper with only rudimentary and understand the experiment?
        \item What are the other, outside factors that may affect the results?
        \item How do you process your data?
        \item How do you quantify the error?
        \item How do you quantify the efficacy?
    \end{itemize}

    \subsection{Subsection of Theory}

    \subsubsection{Subsubsection of Theory}


    \section{Methodology} \label{methodology}
    \subsection{Overview of Experimental Procedure}

    \subsection{Data Processing}
    In this section, you should go into detail of how the formulae introduced in Section~\ref{theory}.
    Basically, you should walk through what you will be putting into Excel.
    All explicit calculations should be included in the Appendix.


    \section{Results}
    \begin{table}[ht]
        \centering
        \caption{Random Data Table}
        \begin{tabular}{ccc}
            \toprule
            \textbf{Column 1} & \textbf{Column 2} & \textbf{Column 3} \\
            \midrule
            Row 1, Cell 1     & Row 1, Cell 2     & Row 1, Cell 3     \\
            Row 2, Cell 1     & Row 2, Cell 2     & Row 2, Cell 3     \\
            Row 3, Cell 1     & Row 3, Cell 2     & Row 3, Cell 3     \\
            \bottomrule
        \end{tabular}
    \end{table}


    \section{Conclusion}
    Wrap up the paper.
    Additional questions:
    \begin{itemize}
        \item How do the results relate to the real world?
        \item What are the next steps?
        \item How can the experiment be improved?
    \end{itemize}


    \appendix


    \section{First Appendix} \label{sec:-first-appendix}


    \section{Code} \label{sec:minted-code}
    \inputminted[linenos, bgcolor=LightGray, fontsize=\footnotesize]{python}{EXAMPLE_Code.py}


% Bibliography (if using BibTeX)
    \bibliography{references}
    \bibliographystyle{ieeetr}


\end{document}
