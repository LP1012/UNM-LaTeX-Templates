\documentclass{413L}


%%%%%%%%%%%%%%%%%%%%%%%%%%%%%%%%%%%
\title{\Large AGN-\#: EXAMPLE}\vspace{0.5cm}
\author{Liam Pohlmann$^{\dagger}$}

\institute{
    $^{\dagger}$Undergraduate, Department of Nuclear Engineering, University of New Mexico, Albuquerque, NM
}

%% Optional disclaimer: remove this command to hide
%\disclaimer{Notice: this manuscript is a work of fiction. Any resemblance to
%actual articles, living or dead, is purely coincidental.}

%%%% packages and definitions (optional)
\usepackage{graphicx}   % allows inclusion of graphics
\usepackage{booktabs}   % nice rules (thick lines) for tables
\usepackage{microtype}  % improves typography for PDF
\usepackage{physics}    % holds useful math operators
\usepackage{setspace}
\usepackage{isotope}
\usepackage[version=4]{mhchem} % \ce{CO2 + C -> 2 CO}
\usepackage{units}
\usepackage[noabbrev,capitalize]{cleveref}  % greatly simplify citations
\usepackage[labelfont=bf]{caption}          % bold the captions
\usepackage[symbol]{footmisc}               % use symbols for footnotes when desired
\usepackage{subcaption}
\usepackage{textcomp}
\usepackage{mathtools}
\usepackage{amssymb}
\usepackage{gensymb}

\onehalfspace

\begin{document}
    \singlespace

%    \tableofcontents
%   %%%%%%%%%%%%%%%%%%%%%%%%%%%%%%%%%%%%%%%%%%%%%%%%%%%%%%%%%%%%%%%%%%%%%%%%%%%%%%%

    \begin{abstract}
    The abstract should briefly summarize the entire report, ideally in \textbf{150–250 words}. Include:
    \begin{enumerate}
        \item The purpose or objective of the experiment
        \item The main methods or procedures used
        \item The most significant results
        \item A concise conclusion or implication of the findings
    \end{enumerate}
    Avoid citations, figures, or equations. Keep it self-contained and understandable to someone who hasn’t read the rest of the paper.
\end{abstract}

\section{Introduction}
The Introduction sets the stage for the reader. Include:
\begin{enumerate}
    \item Background on the topic and its relevance
    \item A brief review of any necessary literature or foundational concepts
    \item The motivation for the experiment
    \item A clear statement of the objective or hypothesis
    \item A sentence or two outlining the structure of the report
\end{enumerate}

\section{Theory}
This section explains the scientific principles behind the experiment. Use \textbf{equations and diagrams as needed}.

\begin{enumerate}
    \item Define all key variables and assumptions
    \item Discuss relevant laws, formulas, or models
    \item Include derivations or simplifications as necessary
    \item Use the following example to format equations with the \texttt{split} environment:
\end{enumerate}

\begin{equation}
\label{eq:fourier-flux}
\begin{split}
    q'' &= -k \frac{dT}{dx} \\
    Q &= \int_A q'' \, dA = -k A \frac{\Delta T}{L}
\end{split}
\end{equation}

\begin{enumerate}
    \item[5.] Explain each step and define all symbols (\(q''\): heat flux, \(k\): thermal conductivity, etc.)
\end{enumerate}

Note that to reference an equation, use \cref{eq:fourier-flux} so that the formatting looks good.
For citations, use \verb|\cite{key}|, example: ~\cite{EXAMPLE}, and be sure to run \texttt{BibTeX} after running \texttt{PDFLaTeX} once.

\section{Methods and Procedures}
Describe how the experiment was conducted in \textbf{step-by-step detail}.

\subsection{Experiment}
\begin{enumerate}
    \item Outline materials used (equipment, sensors, etc.)
    \item Describe the experimental setup, ideally with a diagram
    \item Present step-by-step procedures
    \item Mention any calibration or safety steps
\end{enumerate}

\subsection{Data Processing}
\begin{enumerate}
    \item Explain how raw data was converted into usable results
    \item Mention any software or analysis tools used
    \item Describe error analysis techniques or uncertainty estimation
    \item Show sample calculations, if applicable
\end{enumerate}

\section{Results and Analyses}
Present the data and analyze what it means. Length: 

\begin{enumerate}
    \item Include tables, graphs, and figures as needed
    \item Describe trends, patterns, or anomalies
    \item Compare with theoretical expectations or literature
    \item Include quantitative error analysis
    \item Interpret what the results suggest in terms of the experiment’s goals
\end{enumerate}

\section{Conclusion}
Summarize findings and reflect on implications. 

\begin{enumerate}
    \item Recap the key results and what they mean
    \item State whether the hypothesis was supported
    \item Discuss sources of error and limitations
    \item Suggest improvements or future work
    \item Avoid repeating data; focus on broader takeaways
\end{enumerate}





    %%%%%%%%%%%%%%%%%%%%%%%%%%%%%%%%%%%%%%%%%%%%%%%%%%%%%%%%%%%%%%%%%%%%%%%%%%%%%%%%
    \bibliographystyle{agn}
    \bibliography{references}

\end{document}

